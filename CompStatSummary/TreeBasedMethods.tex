\section{Tree based methods}
tree assumes model of following form: 
$f(x) = \sum_{m=1}^M c_m*1_{(X \in R_m)}$, this is better than linear if highly complex relationship between features and response. If you fit a linear model regressing the response y on the indicator variables $1_{(X \in R_m)}$ it will give you the same fitted values as the tree. The covariates $1_{(X \in R_m)}$ are all orthogonal.
\textbf{Recursive binary splitting: }
Idea: top-down greedy approach that always takes the best predictor to split some box in two. Algortihm:
1) find the best cutting point for each predictor
2) choose predictor which with best cutting point minimizes error ($\sum_{i:x \in R_1}(y_i-\hat y_{R_1})^2+\sum_{i:x \in R_2}(y_i-\hat y_{R_2})^2)$
3) repeat until stopping condition, eg. each box contains < 5 elements
\textbf{pruning: } Recursive binary splitting overfitts to training data, so we want to keep the tree simpler to have a low test error. How: we penalize large subtrees like lasso: $\sum_{m=1}{|T|}\sum_{i: x_i \in R_m}(y_i - \hat y_{R_m})^2 + \alpha|T|$. The right $\alpha$ parameter can be found via crossvalidation. 
\textbf{classification trees: } we grow these trees very similarily to regression trees, but can't take the RSS anymore. Now we want to measure 'node purity', for that we take Gini index: $G = \sum_{k=1}^k \hat p_{mk}(1-\hat p_{mk})$ where $p_{mk}$ is the proportion in the m'th region of the kth class.

\textbf{fit tree in R:}\begin{lstlisting}[language=R] library(tree)
#we want to consider classification, so turn sales into a binary variable
High=ifelse(Carseats$Sales<=8,"No","Yes")
cbind(Carseats$Sales, High)
Carseats = data.frame(Carseats,High)
tree.carseats = tree(High ~ .-Sales, Carseats)}
tree(formula = Sales ~., data = Carseats.train)}#regression tree
prune.10 <- prune.misclass(tree.carseats, best=10) #pruning tree
cv.boston=cv.tree(tree.boston) #cv on tree
plot(cv.boston$size,cv.boston$dev,type='b')
best.size <- cv.boston$size[which.min (cv.boston$dev)]
#pruning to best size (found by cv): 
 pruned.boston <- prune.tree(tree.boston, best = best.size)
yhat=predict(tree.boston, newdata=Boston[-train,]) #pred with tree \end{lstlisting}