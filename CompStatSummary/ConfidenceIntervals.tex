\section{Confidence intervals}
\textbf{CI in R:} 
\begin{lstlisting}[language=R]
confint(fit)
\end{lstlisting}
\textbf{95\% CI for intercept by hand:} 
\begin{lstlisting}[language=R]
n <- nrow(thuesen)
coef(fit)[1] - qt(.975, n-2)*.11748
coef(fit)[1] + qt(.975, n-2)*.11748
\end{lstlisting}
\textbf{Confidence interval for $E(y_0)$:}
\begin{lstlisting}[language=R]
quant <- qt(.975,n-2))
sigma.hat <- sqrt(sum((fit$resid)^2)/(n-2))
X <- as.matrix(cbind(1,thuesen[,1]))
XtXi <- solve(t(X) %*% X)
x00 <- as.matrix(c(1,x0),nrow=2)
se <- sigma.hat * sqrt( t(x00) %*% XtXi %*% x00)
lower <- fitted - quant*se
upper <- fitted + quant*se
predict(fit, pred.frame, level=.95, interval="c")#make sure that naming when doing pred.frame = data.frame(..) is same as in lm!!\end{lstlisting}
\textbf{Prediction interval ( confidence interval for $y_0$):}
\begin{lstlisting}[language = R]
se <- sigma.hat * sqrt( 1 + t(x00) %*% XtXi %*% x00)
lower <- fitted - quant*se
upper <- fitted + quant*se
predict(fit, pred.frame, level=.95, interval="p")#or automatically
\end{lstlisting}